\section{Modello Circuitale}
Nel corso del nostro studio dell'elettrotecnica, ci troveremo ad analizzare quelli che sono i circuiti elettrici; per fare ciò ci serviamo del \textbf{modello circuitale}.
Questo modello è una astrazione semplificata della teoria dell'elettromagnetismo di Maxwell \footnote{vedi Leggi di Maxwell per maggiori informazioni.} che ci permette di analizzare i componenti e le interazioni tra di loro, attraverso i loro punti di connessione, indicati come \textbf{terminali}.\\
Nel modello circuitale, i componenti fisici vengono rappresentati come \bluebf{bipoli} che vengono caratterizzati dalle due proprietà essenziali: \textbf{corrente} e \textbf{tensione}. \\
Nei paragrafi seguenti verranno viste come queste proprietà definiscono il singolo bipolo e le leggi che le governano.

\subsection{Grandezze Circuitali}
Come detto in precedenza, i bipoli che fanno parte del circuito rispondono a determinate leggi e grandezze: le due grandezze principali sono corrente e tensione.\\
Prima di definire circuiti, bipoli e le loro caratteristiche, è dovere definire il materiale di cui sono composti e come questi si comportano in determinate condizioni; a questo scopo definiamo 

\paragraph{materiali conduttori} sono materiali in grado di far scorrere corrente elettrica al loro interno. Ovvero sono materiali, o meglio elementi, strutturati in modo da avere i cosiddetti \textit{elettroni liberi}.
\paragraph{materiali isolanti} sono materiali che si oppongono al movimento di correnti elettriche al loro interno; sono considerati elementi stabili, e quindi non hanno elettroni liberi di muoversi nel materiale.

\subsection{Carica Elettrica}

\subsection{Corrente}

\subsection{Tensione}


\section{Generatori di Forza ElettroMotrice (F.E.M.)}

\section{}